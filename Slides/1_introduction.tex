\documentclass[nopagenumber,9pt]{beamer}

\mode<presentation> {
  \usetheme[]{CambridgeUS}
  %\useoutertheme{shadow}
  \setbeamercovered{transparent}
  \usecolortheme{seahorse}
%\usecolortheme{sidebartab}
%  \usefonttheme{structurebold}
  \useinnertheme{default}
\useinnertheme{rounded}
}
\usepackage{nicefrac}
\RequirePackage{amsmath,amsfonts,amsthm}

\usepackage{float}
\usepackage[english]{babel}
\usepackage{amsmath}
\usepackage[utf8]{inputenc}
\usepackage{times}
\usepackage{url}
\usepackage[T1]{fontenc}
%\usepackage{multirow}
\usepackage{color}
\newcommand{\mb}[1]{\mathbf{#1}}
\usepackage{graphicx}
\graphicspath{{./figure/}}

\usepackage[ruled,vlined]{algorithm2e}
\usepackage{xcolor,colortbl}
\usepackage{rotating}
\usepackage{multirow}

\usepackage{tikz}
\usepackage{ulem}



\definecolor{dgreen}{RGB}{139,172,100}

\hypersetup{
  colorlinks = true,
  linkcolor = black
}
\makeatletter
\let\@mycite\@cite
\def\@cite#1#2{{\hypersetup{linkcolor=dgreen}[{#1\if@tempswa , #2\fi}]}}
\makeatother

\newtheorem{proposition}{Proposition}
% \newcommand{\I}{\mathbb{I}}
% \newcommand{\E}{\mathbb{E}}
% \renewcommand{\P}{\mathbb{P}}
% \newcommand{\R}{\mathbb{R}}
% \newcommand{\bs}{\boldsymbol}
% \newcommand{\bbeta}{\boldsymbol{\beta}}
% \newcommand{\balpha}{\boldsymbol{\alpha}}
% \newcommand{\btheta}{\boldsymbol{\theta}}
% \newcommand{\bY}{\mathbf{Y}}
% \newcommand{\bX}{\mathbf{X}}
% \newcommand{\bZ}{\mathbf{Z}}
% \newcommand{\by}{\mathbf{y}}
% \newcommand{\bz}{\mathbf{z}}
% \newcommand{\ba}{\mathbf{a}}
% \newcommand{\bx}{\mathbf{x}}
% \newcommand{\bh}{\mathbf{h}}
% \newcommand{\bb}{\mathbf{b}}
% \newcommand{\bB}{\mathbf{B}}
% \newcommand{\bM}{\mathbf{M}}
% \newcommand{\bphi}{\boldsymbol{\phi}}
% \newcommand{\bpsi}{\boldsymbol{\psi}}
% \newcommand{\bpi}{\boldsymbol{\pi}}
% \newcommand{\btau}{\boldsymbol{\tau}}
% \newcommand{\Ecal}{\mathcal{E}}
% \newcommand{\GP}{\mathcal{GP}}
% \newcommand{\bxi}{\boldsymbol{\xi}}
% \newcommand{\brho}{\boldsymbol{\rho}}
% \newcommand{\bgamma}{\boldsymbol{\gamma}}
% \newcommand{\bsigma}{\boldsymbol{\sigma}}
%
%
% \newcommand{\nf}{n_e}
%
%
% %%%macros
% \newcommand{\yexp}{y^{e}}
% \newcommand{\byexp}{\by^{e}}
% \newcommand{\yexpi}{\yexp_i}
% \newcommand{\bxexp}{\bx^e}
%
%
% \newcommand{\Df}{D^e}
% \newcommand{\byc}{\by^c}
%%%% convention
% phenomene reel zeta

\usepackage{macros}


\title[Introduction]{Calibration of computer models}

%titre premiere page

\subtitle{Introduction}

\author[P. Barbillon]{ Pierre \textsc{Barbillon}}
\bigskip

\date{Fall 2023}

\subject{Séminaire}



\AtBeginSection[] {
 \begin{frame}<beamer>
   \frametitle{Outline}
   \tableofcontents[currentsection]
  \end{frame}
}

\AtBeginSubsection[] {
\begin{frame}<beamer>
   \frametitle{Outline}
   \tableofcontents[currentsection,currentsubsection]
 \end{frame}
}


\begin{document}

\begin{frame}
\titlepage
%\includegraphics[scale=.12]{AgroParisTech_-_logo.PNG}
%\vspace{-1.5cm}
%\begin{flushright}
% \includegraphics[scale=.1]{Logotype-INRA-transparent.png}
% \end{flushright}
\vspace{-1cm}
\centering
\begin{tabular}{ccc}
 \includegraphics[scale=.08]{LogoUPSaclay.jpg}&
  \includegraphics[scale=1.3]{agrologo.png}&
   \includegraphics[scale=.1]{LogoINRAE.jpg}
\end{tabular}


\end{frame}



\begin{frame}
\frametitle{Uncertainty Quantification / Model Uncertainty}

{\centering
\includegraphics[scale=.25]{SchemaUQ.pdf}}


In this course, we will focus on calibration and validation.

\bigskip



\end{frame}



\begin{frame}
\frametitle{Calibration of a computer code} 
 \textbf{Computer experiments:}
 \\
 \medskip
Computer model (simulator)  $(\bx,\btheta)\mapsto f(\bx,\btheta)\in \mathbb{R}^s $ where\\
\medskip
\begin{itemize}
  \item \textbf{physical parameters}: $\bx\in  \mathbb{X}\subset\mathbb{R}^p$ observable and often controllable inputs
  \medskip
  \item \textbf{simulator parameters}: $\btheta\in \Theta\subset\mathbb{R}^d$ non-observable parameters, required to run the simulator.
  \\
  \smallskip
  2 types:
  \begin{itemize}
   \item ``calibration parameters'': physical meaning but unknown, necessary to make the code mimic the reality,  
   \item ``tuning parameters'': no physical interpretation.
  \end{itemize}
\end{itemize}

  \bigskip
  
  \textbf{Goal:}\\
  Calibrate the code: finding ``best'' or ``true'' $\btheta$ from real observations / field data 
  (provided by physical experiments):
  $$\byexp=\{\yexp_1=\zeta(\bxexp_1),\ldots,\yexp_n=\zeta(\bxexp_n)\}\,,$$

  where $\zeta$ is the real physical phenomenon.
%   \textbf{Observations / Data:}
%   For different inputs: $\bx_1,\ldots,\bx_n$\\
  
  \end{frame}

  
  
  \begin{frame}
 \frametitle{Validation}
 \begin{itemize}
  \item Validation (rather than verification) is considered,
  \item[]
  \item Does the computer simulator correspond to field data?
  
  $$\exists \btheta^*?, \ \textrm{s.t.},\ \forall \bx,\quad f(\bx,\btheta^*)\approx y(\bx)$$
  
  \item[]
  
  
  \item This question is related with intended use of the simulator: range of $\bx$, required precision... 
  
  \item[]
  
%   \item The validation of the computer simulator depends on the known or unknown precision of the field data
%   
%   \item[]
  
  \item Biased computer model, no setting of calibrated parameters leads to outputs close to field data\\
  $\Rightarrow$ {\color{red} discrepancy}.\\
%   What is the meaning of validation
%   in that context?  
%   
  
  \item[]
  
 \item Do we want to validate the computer model itself or the computer model with the bias / discrepancy correction?
 \end{itemize}

\end{frame}



  \begin{frame}
  \frametitle{Framework}
  KOH framework chosen for this course: \cite{kennedy2001,higdon2004,bayarri2007}.
 

  \bigskip
  
  History Matching not consider but may be relevant depending on the objective \cite{craig1997pressure,vernon2010galaxy, Boukouvalas2014Calibration, Andrianakis2017efficient}
   
  \end{frame}


\begin{frame}
 \frametitle{Deterministic or stochastic simulator}
 
 
 In most contributions, $f$ is considered as deterministic.
 
 \bigskip
 
 But some recent work considered $f$ as stochastic (\cite{baker}):
 
 
 \begin{itemize}
  \item $f$ uses stochastic approximations (MC,...) but the modeled phenomenon is deterministic $\zeta$...
  
  \medskip
  
  \item $f$ models a stochastic phenomenon $\zeta(\cdot)$ is stochastic.
  
 
 \end{itemize}

 
 
 
 
\end{frame}
%
% \begin{frame}
%  \frametitle{Example}
%  to be done with calico
% \end{frame}
%


\begin{frame}
 \frametitle{Notations}
 
 \begin{itemize}
  \item $\zeta(\bx)$ real physical or biological phenomenon,
  \item $f(\bx,\btheta)$ numerical code/model with $\bx$ bservable or controllable input variable, $\btheta$ model parameter (no counterpart in the real phenomenon.
  \item DoNE: Design of Numerical Experiments: $\Dc=\{(\bx_1,\btheta_1),\ldots,(\bx_N,\btheta_N)\}$  %(space-filling or optimize with respect to a given goal: rare event, global optimization)
with corresponding evaluations of the computer model (time-consuming):
$$\byc=f(\Dc)=\{f(\bx_1,\btheta_1),\ldots,f(\bx_N,\btheta_N)\}\,.$$


 \item DoFE: Design of Field Experiments: $\Dexp=\{\bxexp_1,\ldots,\bxexp_{\nexp}\}$ with corresponding noisy observation of $\zeta$:$$\byexp=\{y_1=\zeta(\bxexp_1)+\epsilon_1,\ldots,y_n=\zeta(\bxexp_{\nexp})+\epsilon_{\nexp}\}\,.$$
 \end{itemize}

 
\end{frame}



\begin{frame}
 \frametitle{Outline}
 
 \begin{enumerate}
  \item Calibration KOH and extensions
  
  \item Sequential design of experiments
  
  \item Focus on the discrepancy function and validation
  
  \item Extensions to calibration of stochastic simulator
  
 \end{enumerate}

 
\end{frame}



\bibliographystyle{apalike}
\bibliography{biblio}


\end{document}
